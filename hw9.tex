\documentclass[11pt]{article}
\title{\textbf{MATH 444 HW 9}}
\author{Nathaniel Murphy}
\date{}

\usepackage{a4wide}
\usepackage{amssymb}
\usepackage{amsmath}
\usepackage{mathtools}
\usepackage{mathrsfs}

\begin{document}
\maketitle

\section*{5.1.12}
Suppose that $f(x)\neq 0$ for $x \in \mathbb{R} \setminus \mathbb{Q}$. But, in 4.1.15, we explored the function $g:\mathbb{R}\rightarrow\mathbb{R}$, where $g(x)=x$ if $x$ is rational and $g(x)=c$ if $x$ is irrational. We proved that if $c\neq 0$, that there exists no limit at point $c$. We can translate this to this problem by seeing that if $f(x)\neq 0$ for all irrationals, that there exists no limit of $f$ at $x$, and therefore, $f$ is not continuous at point $x$. This is a contradiction because $f$ is given to be continuous, so therefore $f(x)=0$ for every $x\in\mathbb{R}$.

\section*{5.1.13}
$g$ will be continuous only where $2x=x+3$, which is $x=3$.

\end{document}